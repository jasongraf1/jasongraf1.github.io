\documentclass[11pt,]{article}
\usepackage[sc, osf]{mathpazo}
\usepackage{amssymb,amsmath}
\usepackage{ifxetex,ifluatex}
\usepackage{fixltx2e} % provides \textsubscript
\ifnum 0\ifxetex 1\fi\ifluatex 1\fi=0 % if pdftex
  \usepackage[T1]{fontenc}
  \usepackage[utf8]{inputenc}
\else % if luatex or xelatex
  \ifxetex
    \usepackage{mathspec}
  \else
    \usepackage{fontspec}
  \fi
  \defaultfontfeatures{Ligatures=TeX,Scale=MatchLowercase}
\fi
% use upquote if available, for straight quotes in verbatim environments
\IfFileExists{upquote.sty}{\usepackage{upquote}}{}
% use microtype if available
\IfFileExists{microtype.sty}{%
\usepackage{microtype}
\UseMicrotypeSet[protrusion]{basicmath} % disable protrusion for tt fonts
}{}
\usepackage[margin=1in]{geometry}




\setlength{\emergencystretch}{3em}  % prevent overfull lines
\providecommand{\tightlist}{%
  \setlength{\itemsep}{0pt}\setlength{\parskip}{0pt}}
\setcounter{secnumdepth}{0}
% Redefines (sub)paragraphs to behave more like sections
\ifx\paragraph\undefined\else
\let\oldparagraph\paragraph
\renewcommand{\paragraph}[1]{\oldparagraph{#1}\mbox{}}
\fi
\ifx\subparagraph\undefined\else
\let\oldsubparagraph\subparagraph
\renewcommand{\subparagraph}[1]{\oldsubparagraph{#1}\mbox{}}
\fi

% Now begins the stuff that I added.
% ----------------------------------

% Custom section fonts
\usepackage{sectsty}
\sectionfont{\rmfamily\mdseries\large\bf}
\subsectionfont{\rmfamily\mdseries\normalsize\itshape}


% Make lists without bullets
%\renewenvironment{itemize}{
%  \begin{list}{}{
%    \setlength{\leftmargin}{1.5em}
%  }
%}{
%  \end{list}
%}


% Make parskips rather than indent with lists.
\usepackage{parskip}
\usepackage{titlesec}
\titlespacing\section{0pt}{12pt plus 4pt minus 2pt}{4pt plus 2pt minus 2pt}
\titlespacing\subsection{0pt}{12pt plus 4pt minus 2pt}{4pt plus 2pt minus 2pt}

% Use fontawesome. Note: you'll need TeXLive 2015. Update.
\usepackage{fontawesome}

% Fancyhdr, as I tend to do with these personal documents.
\usepackage{fancyhdr,lastpage}
\pagestyle{fancy}
\renewcommand{\headrulewidth}{0.0pt}
\renewcommand{\footrulewidth}{0.0pt}
\lhead{}
\chead{}
\rhead{}
\lfoot{
\cfoot{\scriptsize  Jason Grafmiller - CV - orcid.org/0000-0002-9577-6885 }}
\rfoot{\scriptsize \thepage/{\hypersetup{linkcolor=black}\pageref{LastPage}}}

% Always load hyperref last.
\usepackage{hyperref}
\PassOptionsToPackage{usenames,dvipsnames}{color} % color is loaded by hyperref

\hypersetup{unicode=true,
            pdftitle={Jason Grafmiller:  CV (Curriculum Vitae)},
            pdfauthor={Jason Grafmiller},
            pdfkeywords={RMarkdown, academic CV, template},
            colorlinks=true,
            linkcolor=blue,
            citecolor=Blue,
            urlcolor=blue,
            breaklinks=true, bookmarks=true}
\urlstyle{same}  % don't use monospace font for urls

\begin{document}


\centerline{\huge \bf Jason Grafmiller}

\vspace{2 mm}

\hrule

\vspace{2 mm}

\moveleft.5\hoffset\centerline{Lecturer}
\moveleft.5\hoffset\centerline{3 Elms Road · Birmingham · United Kingdom}
\moveleft.5\hoffset\centerline{ \faEnvelopeO \hspace{1 mm} \href{mailto:}{\tt \href{mailto:j.grafmiller@bham.ac.uk}{\nolinkurl{j.grafmiller@bham.ac.uk}}} \hspace{1 mm}  \faPhone \hspace{1 mm}  +44 121 414 7017  \hspace{1 mm}  \faGithub \hspace{1 mm} \href{http://github.com/jasongraf1}{\tt jasongraf1} \hspace{1 mm}    \faGlobe \hspace{1 mm} \href{http://xxx}{\tt xxx}    | \emph{Updated:} \today}

\vspace{2 mm}

\hrule


\section{EDUCATION}\label{education}

\emph{Stanford University}, Ph.D.~Linguistics \hfill 2013\\
\emph{The Ohio State University}, B.A. Linguistics \hfill 2006

\section{EMPLOYMENT}\label{employment}

\emph{English Language and Applied Linguistics, University of
Birmingham}

\begin{quote}
Lecturer in Corpus-based Sociolinguistics \hfill 2017--present
\end{quote}

\emph{Quantitative Lexicology and Variational Linguistics, KU Leuven}

\begin{quote}
Post-doctoral fellow \hfill 2013--17
\end{quote}

\section{TEACHING}\label{teaching}

\emph{Sounds, Structures and Words.} UG year 1 \hfill UBham 2017-2018

\emph{Research Skills in English Language.} UG year 2 \hfill UBham
2017-2018

\emph{English as an International Language.} MA level \hfill UBham
2017-2018

\emph{Social and Psychological Aspects of Language.} MA level
\hfill UBham 2017-2018

\emph{Methods of Corpus Linguistics.} MA level \hfill KU Leuven
2016-2017

\begin{quote}
Introduction to methods and techniques for dealing with the large
collections of usage data found in linguistic corpora, and discussion of
how these techniques contribute to linguistic theory development.
Methods include: collocational/collostructional analysis, logistic
regression, factor and correspondence analsys. MA level
\end{quote}

\emph{Introduction to modeling linguistic variation with R.} \hfill 2016

\begin{quote}
Invited workshop tutorial on variationist methods using R. University of
Uppsala, Nov 1--2
\end{quote}

\emph{English Linguistics: Special Topics.} MA level \hfill KU Leuven
2015

\begin{quote}
Seminar exploring the theoretical and empirical approaches to the study
of argument realization in English, with special focus on English
psychological verbs (``fear'', ``frighten''). In the course we consider
what makes psych-verbs special as a class, and why (and in what ways)
they challenge different theories of argument expression, event
structure and the syntax-semantics interface.
\end{quote}

\emph{English Linguistics: Special Topics.} MA level \hfill KU Leuven
2014

\begin{quote}
Seminar on the use of different methods for the study of syntactic
variation within and across language varieties. We introduce basic
concepts in variationist sociolinguistics and statistical tools, and
discuss their application to the investigation of several grammatical
phenomena in the recent literature.
\end{quote}

\emph{Quantitative Methods in Linguistics.} MA level \hfill Stanford
2013

\begin{quote}
Introduction to methods for collecting and analyzing quantitative
linguistic data, with a focus on the use of corpora in exploring
theoretical questions in various area of linguistics. Topics will
include the access and retrieval of corpus data, data annotation, and
statistical modeling, using a variety of software tools (e.g.~Python,
R).
\end{quote}

\subsection{As Teaching Assistant}\label{as-teaching-assistant}

\emph{Introduction to Syntax (HPSG).} Stanford University. Instructor:
Hanna Rohde

\emph{Language in Society.} Stanford University. Instructor: Adam Hodges

\emph{Introduction to Linguistics.} Stanford University. Instructors:
Penny Eckert and Ivan Sag

\section{PUBLICATIONS}\label{publications}

\subsection{Journal articles}\label{journal-articles}

\begin{enumerate}
\def\labelenumi{\arabic{enumi}.}
\setcounter{enumi}{2016}
\item
  Röthlisberger, Melanie, Jason Grafmiller \& Benedikt Szmrecsanyi.
  Cognitive indigenization effects in the English dative alternation.
  \emph{Cognitive Linguistics}. DOI: 10.1515/cog-2016-0051.
\item
  Szmrecsanyi, Benedikt, Jason Grafmiller, Joan Bresnan, Anette
  Rosenbach, Sali Tagliamonte \& Simon Todd. Spoken syntax in a
  comparative perspective: the dative and genitive alternation in
  varieties of English. \emph{Glossa}. DOI: 10.5334/gjgl.310
\item
  Heller, Benedikt, Benedikt Szmrecsanyi \& Jason Grafmiller. Stability
  and fluidity in syntactic variation world-wide: The genitive
  alternation across varieties of English. \emph{Journal of English
  Linguistics} 45(1). DOI: 10.1177/0075424216685405.
\item
  Grafmiller, Jason, Benedikt Szmrecsanyi \& Lars Hinrichs. Restricting
  the restrictive relativizer: Constraints on subject and non-subject
  English relative clauses. \emph{Corpus Linguistics and Linguistic
  Theory}. DOI: 10.1515/cllt-2016-0015.
\item
  Szmrecsanyi, Benedikt, Jason Grafmiller, Benedikt Heller \& Melanie
  Röthlisberger. Around the world in three alternations: Modeling
  syntactic variation in global varieties of English. \emph{English
  World Wide} 37(2). DOI: 10.1075/eww.37.2.01szm.
\item
  Grafmiller, Jason. Variation in English genitives across modality and
  genres. \emph{English Language and Linguistics}, 18(3): 471-496. DOI:
  10.1017/S1360674314000136.
\end{enumerate}

\subsection{Book chapters}\label{book-chapters}

\begin{enumerate}
\def\labelenumi{\arabic{enumi}.}
\setcounter{enumi}{2014}
\item
  Shih, Stephanie, Jason Grafmiller, Richard Futrell \& Joan Bresnan.
  Rhythm's role in genitive and dative construction choice in English.
  In R. Vogel and R. van de Vijver (eds) \emph{Rhythm in Phonetics,
  Grammar, and Cognition}. De Gruyter, 207-234.
\item
  Levin, Beth \& Jason Grafmiller. Do you always fear what frightens
  you? In T. H. King and V. de Paiva (eds) \emph{From Quirky Case to
  Representing Space}. CSLI Publications, Stanford, 21-32.
\end{enumerate}

\section{PRESENTATIONS}\label{presentations}

Grafmiller, Jason. 2016. Methods in Corpus-based Variationist
Linguistics. Invited talk. University of Uppsala, Uppsala, Sweden. Oct
31.

Grafmiller, Jason. 2016. Reconsidering the locative syntax of
experiencers in English. International Society for the Linguistics of
English 4, Poznan, Poland, September 21.

Grafmiller, Jason. 2016. Mapping out particle placement around the
globe: A corpus study of indigenization patterns. International Society
for the Linguistics of English 4, Poznan, Poland, September 19.

Grafmiller, Jason, Benedikt Heller, Melanie R¨othlisberger \& Benedikt
Szmrecsanyi. 2016. Syntactic variation and probabilistic indigenization
in World Englishes. New Ways of Analyzing Syntactic Variation 2, Ghent,
Belgium, May 19.

Grafmiller, Jason, Benedikt Heller, Melanie R¨othlisberger \& Benedikt
Szmrecsanyi. 2016. Syntactic variation as a measure of probabilistic
indigenization in global varieties of English. The 90th Annual Meeting
of the Linguistic Society of America, Washington D.C., January 10.

Grafmiller, Jason \& Melanie R¨othlisberger. 2015. Syntactic
alternations, schematization, and collostructional diversity in world
Englishes. 21st conference of the International Association for World
Englishes, Istanbul. October 9.

Röthlisberger, Melanie, Jason Grafmiller, Benedikt Heller \& Benedikt
Szmrecsanyi. 2015. What choice(s) do we have? Processing and contextual
constraints on syntactic variation across the globe. 21st conference of
the International Association for World Englishes, Istanbul. October 9.

Grafmiller, Jason. 2015. Deviant diachrony: Exploring new methods for
analyzing language change. New Developments in the Quantitative Study of
Languages, Helsinki. August 29.

Grafmiller, Jason. 2015. Transitivity and construal in English emotion
verbs: A quantitative investigation. 13th International Cognitive
Linguistics Conference, Newcastle. July 21.

Grafmiller, Jason. 2015. On the ole of agency and causation in the
semantics of emotion verbs: An empirical investigation. 23rd annual
meeting of the European Society for Philosophy and Psychology, Tartu,
Estonia. July 17.

Szmrecsanyi, Benedikt, Jason Grafmiller, Melanie R¨othlisberger \&
Benedikt Heller. 2015. Probabilistic variation in a comparative
perspective: the grammar of varieties of English. ICLaVE 8, Leipzig. May
27.

Grafmiller, Jason. 2014. Modeling verb meanings: An experimental and
corpusbased investigation of English emotion verbs. Invited talk.
University of Bamberg, Bamberg, Germany. December 18.

Grafmiller, Jason, Benedikt Heller, Melanie R¨othlisberger \& Benedikt
Szmrecsanyi. 2014. Exploring probabilistic grammar(s) in Englishes
around the world. ISLE 3, Zurich. August.

Grafmiller, Jason, Benedikt Heller, Melanie R¨othlisberger \& Benedikt
Szmrecsanyi. 2014. Exploring probabilistic grammar(s) in Englishes
around the world. ICAME 35, Nottingham. May 1.

Grafmiller, Jason. 2013. Modeling verb meaning with corpus data: A
usage-based investigation of argument realization in English
psych-verbs. American Association for Corpus Linguistics, San Diego, CA.
January 18.

Grafmiller, Jason. 2013. Object-Experiencer verbs as true transitive
verbs. Poster at the 87th Annual Meeting of the Linguistic Society of
America, Boston, MA. January 5.

\textbf{2012. Grafmiller, Jason.} On the status of agentivity in
Experiencer-object verbs. Agents and causes: Interdisciplinary aspects
in mind, language and culture, Bielefeld, Germany. March 22.

\textbf{2012. Grafmiller, Jason.} Agency is inferred in causal events:
Evidence from ObjectExperiencer verbs. Poster at \emph{The 86th Annual
Meeting of the Linguistic Society of America}, Portland, OR. January 6.

\textbf{2011. Grafmiller, Jason \& Stephanie Shih.} New approaches to
end weight. \emph{Variation and Typology: New Trends in Syntactic
Research}, Helsinki. August 26.

\textbf{2011. Grafmiller, Jason.} Non-structural influences on binding
in Finnish possessives. \emph{The 85th Annual Meeting of the Linguistic
Society of America}, Pittsburg, PA. January 7.

\textbf{2011. Shih, Stephanie \& Jason Grafmiller.} Weighing in on end
weight. \emph{The 85th Annual Meeting of the Linguistic Society of
America}, Pittsburg, PA. January 9.

\textbf{2010. Grafmiller, Jason \& Stephanie Shih.} Weighing in on end
weight. \emph{The Development of Syntactic Alternations Workshop},
Stanford University. November 12.

\textbf{2009. Shih, Stephanie, Jason Grafmiller, Richard Futrell, \&
Joan Bresnan.} Rhythm's role in genitive and dative construction choice
in English. \emph{31st Annual Meeting of the Linguistic Association of
Germany (DGfS)}, Osnabr"uck. March.

\textbf{2008. Dmitrieva, Olga, Matthew Adams, Jason Grafmiller, Scott
Grimm, Yuan Zhao, \& Arto Anttila.} Gradient OCP and harmonic alignment
in English phonotactics. \emph{The 82nd Annual Meeting of the Linguistic
Society of America}, Chicago, IL. January.

\section{HONORS \& AWARDS}\label{honors-awards}

\textbf{2011} First Runner-up, Student Abstract Award (``Weighing in on
end weight'', with Stephanie Shih), 85th Linguistic Society of America
meeting

\textbf{2006} Best Undergraduate Research Paper in Linguistics
(``Agreement Conflicts and Coordinate Nouns''). The Ohio State
University

\end{document}
